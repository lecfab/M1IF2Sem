%/!\ /!\ 
%
% PLEASE DO NOT EDIT THIS IF YOU CAME HERE BY MISTAKE !!!!
%

% RTFMN : https://tobi.oetiker.ch/lshort/lshort.pdf

\documentclass{article}
\usepackage{xspace}
\usepackage[utf8]{inputenc}
\usepackage[T1]{fontenc}
\usepackage[english]{babel}
\usepackage{amsmath}
\usepackage{amsthm}
\usepackage{graphicx}
\usepackage{url}
\usepackage{amssymb}
\usepackage{mathrsfs}
\usepackage{amsfonts}
\usepackage{multicol}
\usepackage{stmaryrd}
\usepackage{tikz, pgf}
\usetikzlibrary{arrows,intersections}
\usepackage{libertine}
\usepackage[a4paper,left=2cm,right=2cm,top=2cm,bottom=2cm]{geometry}
\usepackage{dsfont}


\usepackage[linktocpage]{hyperref}

\setlength{\hoffset}{-18pt}         
\setlength{\oddsidemargin}{15pt} % Marge gauche sur pages impaires
\setlength{\evensidemargin}{15pt} % Marge gauche sur pages paires
\setlength{\marginparwidth}{0pt} % Largeur de note dans la marge
\setlength{\textwidth}{481pt} % Largeur de la zone de texte 
\setlength{\marginparsep}{7pt} % Séparation de la marge
\setlength{\topmargin}{0pt} % Pas de marge en haut
\setlength{\headheight}{13pt} % Haut de page
\setlength{\headsep}{10pt} % Entre le haut de page et le texte
\setlength{\footskip}{50pt} % Bas de page + séparation
\setlength{\textheight}{600pt} % Hauteur de la zone de texte 

%\setlength{\hoffset}{-18pt}         
%\setlength{\oddsidemargin}{15pt} % Marge gauche sur pages impaires
%\setlength{\evensidemargin}{15pt} % Marge gauche sur pages paires
%\setlength{\marginparwidth}{0pt} % Largeur de note dans la marge
%\setlength{\textwidth}{481pt} % Largeur de la zone de texte 
%\setlength{\marginparsep}{7pt} % Séparation de la marge
%\setlength{\topmargin}{0pt} % Pas de marge en haut
%\setlength{\headheight}{8pt} % Haut de page
%\setlength{\headsep}{0pt} % Entre le haut de page et le texte
%\setlength{\footskip}{15pt} % Bas de page + séparation
%\setlength{\textheight}{700pt} % Hauteur de la zone de texte 

%\newcommand{\ket}[1]{\ensuremath{|#1\rangle}\xspace}
%\newcommand{\bra}[1]{\ensuremath{\langle #1|}\xspace}

\newtheorem{thm}{Theorem}[section]
\newtheorem{prop}[thm]{Proposition}
\newtheorem{lem}[thm]{Lemma}
\newtheorem{cor}[thm]{Corollary}
\newtheorem{defi}[thm]{Definition}
\newtheorem{ex}[thm]{Example}

\newcommand{\Thm}[3]{\begin{thm}[#1]\label{#2}#3\end{thm}}
\newcommand{\Ex}[3]{\begin{ex}[#1]\label{#2}#3\end{ex}}
\newcommand{\Def}[3]{\begin{defi}[#1]\label{#2}#3\end{defi}}
\newcommand{\Lem}[3]{\begin{lem}[#1]\label{#2}#3\end{lem}}
\newcommand{\Cor}[3]{\begin{cor}[#1]\label{#2}#3\end{cor}}
\newcommand{\Prop}[3]{\begin{prop}[#1]\label{#2}#3\end{prop}}

\newcommand{\hsp}{\hspace{20pt}}
\newcommand{\HRule}{\rule{\linewidth}{0.5mm}}
\newcommand{\R}{\mathbb{R}}
\newcommand{\N}{\mathbb{N}}
\newcommand{\K}{\mathbb{K}}
\newcommand{\Q}{\mathbb{Q}}
\newcommand{\C}{\mathcal{C}}
\newcommand{\X}{\mathcal{X}}
\newcommand{\brakets}[1]{\langle#1\rangle}
\newcommand{\betaar}{\rightarrow_\beta}
\newcommand{\betar}{\twoheadrightarrow_\beta}
\newcommand{\alphaeq}{=_\alpha}

\newcommand{\ind}[1]{\mathds{1}_{#1}}
\renewcommand{\epsilon}{\varepsilon}

\title{Cryptography}
\author{Benoit Libert et Damien Stehlé	}
\date{Spring 2017}



\begin{document}

\maketitle

\tableofcontents

\newpage

\section{Introduction}
Cryptography is the science of information security, not just encryption. In a sense it is not just Information Theory.

The most classical form of encryption must satisfy:
\begin{itemize}
\item a talk to $B$ iver a public channel
\item we don't eavesdropper C to understand the communication.
\end{itemize}

\Ex{}{}{Take a client and a server such that each of them have a Browser with 2 stages:
\begin{itemize}
\item 1: Handshake. They create a common session key, $K$
\item 2: Using $K$, they can encrypt and decrypt the communication.
\end{itemize}

We will see the first step later, and in the following we will shortly deal with the second.
}

We have to deal with:
\begin{itemize}
\item Secret key cryptography: $A$ and $B$ use the same secret key $K$ to communicate
\item Cryptography is not just an encryption:
\begin{itemize}
\item How $A$ can be sure je is really talking to $B$ (use of digital signatures)
\item How can they know the that the communication has not been modified ? (Question of integrity)
\item What if I don't want to reveal my identity ? (Anonymous communication)
\end{itemize} 
\item More complex primitives and properties:
\begin{itemize}
\item Digital cash: should be anonymous, infeasible to fake or duplicate
\item E-voting: $n$ voter choose candidates, majority wins but votes should be anonymous, it must impossible to vote twice, votes must be verified.
\end{itemize}
\end{itemize}

Cryptography is a science on its own. It has its own methodology. Basically, we can sum it up in this few steps:
\begin{itemize}
\item One defines what the protocol has to do
\item One defines the attacker's goal and power. The aim is to give a security model.
\item One describes a protocol realisation
\item One «proves» that, if an «efficient» attacker exists, then some hard problem can be solved (e.g. factoring, computing discrete logarithms).
\end{itemize}

Cryptography often plays with some hard problems. It is a kind of face of complexity theory. It always suppose that protocols' specifications and description are public, the adversary knows how the protocol works, only the key remains private (Kerckhoff's principle).

However cryptography has its limitations:
\begin{itemize}
\item it relies on algorithmic assumptions and does not ensure unconditional security.
\item does not protect against engineering bugs (FREAK attack (2015), attack intercepts https connections and forces client/server to use weak encryption, or LOGJAM attack (2015), client/server forced to use too short keys)
\item some attacks are out-of-model exists: side-channel information (e.g. power consumption may leak the key)
\item backdoors may be correctly introduced in cryptographic implementations
\item security threats may be unrelated to cryptography
\end{itemize}

\section{One-time pad and perfect security}
\Def{Encrytion Scheme}{def:encrysch}{An encryption scheme $(Keygen, Enc, Dec)$ operates over finite sets $(K,P,C)$ with:
\begin{itemize}
\item $Keygen$ chooses $k$ uniformly in $K$
\item $Enc : K\times P \rightarrow C$
\item $Dec : K\times C \rightarrow P$
\end{itemize}
such that $\forall m\in P, k\in K, Dec(k,Enc(k,m))=m$.
\begin{itemize}
\item $m$ and $k$ are independent random variables
\item $Enc$ may be probabilistic but $Dec$ is usually deterministic
\item We assume (in the following) that elements of $K,P,C$ have non-zero probabilities.
\end{itemize}}

\Def{Perfect security (Shannon, 1949)}{def:perfSec}{For any message distribution $\mathcal{M}$ over $P$, \[\forall \bar{m}\in P, \Pr_{m\hookrightarrow\mathcal{M}, k\hookrightarrow U(K)}(m=\bar{m}|Enc(k,m))\]}

\underline{Remark:} The intuition is that giving $E(k,m)$ does not leaks any information about $m$. Formally, $m$ and $Enc(k,m)$ are independent.

\Lem{Shannon}{}{Perfect security implies that $|K|\geq|P|$}
\begin{proof}
By contradiction assume $|K|<|P|$. Then define $P(\bar{c})=\{m\in P|\exists k\in K, m=Dec(k,\bar{c})\}$. Then $|P(\bar{c})\leq|K|$ since $\forall m\in P(\bar{c}),\exists k\in K, m=Dec(k,\bar{c})$.

Then by our assumption $P-P(\bar{c})\neq\emptyset$. Let $\bar{m}$ be in this set. Then $\Pr(m=\bar{m}|\bar{c}=Enc(k,m))=0\neq\Pr(m=\bar{m})$. That is a contradiction with perfect security (ck previous remark).
\end{proof}

\Def{One-time Pad (OPT), Vernam's cipher (patented in 1917)}{def:OTP}{We take $K=\{0,1\}^l=P=C$. $Enc(k,m) = m\oplus k$ and $Dec(k,c)=c\oplus k$}

\Thm{}{thm:OPTPerfect}{Vernam's cipher is perfectly secret}
\begin{proof}
Let $\bar{m}\in P, \bar{c}\in C$ arbitrary. If we have $\Pr(c=\bar{c}|m=\bar{m})=\Pr(c=\bar{c}))=\Pr(c=\bar{c})$ then by multiplying by $\frac{\Pr(m=\bar{m})}{\Pr(c=\bar{c})}$ we get the result. Then let's compute it:
\[\begin{array}{r c l}
\Pr(c=\bar{c}|m=\bar{m}) & = & \Pr(m\oplus k=\bar{c}|m=\bar{m})\\
&=&\Pr(\bar{m}\oplus k \bar{c})\\
&=&\frac{1}{2^l}
\end{array}\]
\[\begin{array}{r c l}
\Pr(c=\bar{c}) & = & \sum_{\bar{m}}\Pr(\bar{c}=\bar{c}|m=\bar{m})\Pr(\bar{m}=m)\\
&=&\frac{1}{2^l}\sum_{\bar{m}}\Pr(\bar{m}=m)\\
&=&\frac{1}{2^l}
\end{array}\]
\end{proof}

The major problems with that are:
\begin{itemize}
\item key is as long as the message
\item we cannot encrypt twice with the same key ($Enc=Dec$)
\item Assumes an unbounded adversary (too powerful)
\end{itemize}

\section{Pseudo-random generators (PRG) and stream ciphers}
\subsection{One-time pad and pseudo-randomness}
\underline{Idea:} use the OPT with pseudo-random bits rather than truly random ones.

\begin{itemize}
\item We want to use a PRG $G:\{0,1\}^s\rightarrow\{0,1\}^n$ with $n\gg s$
\item Input $k\in\{0,1\}^s$, called \underline{seed} and contains all the randomness
\item $G$ is computable in deterministic polynomial time
\item Tempting idea for encryption: $Enc(k,m) = m\oplus G(k)$ and $Dec(c,k) = c\oplus k$ ($m\in\{0,1\}^n$ is longer than $s$).
\end{itemize}
\underline{Question:} Perfect secrecy is impossible. Can we get computational security ?

\Def{Negligible function}{def:negligFunc}{$f:\N\rightarrow\R$ is negligible if $\forall c>0,\exists n_0\in\N, \forall n>n_0,|f(n)|<n^{-c}$}

\Ex{}{}{$2^{-n}$, $n^{-\log n}$... are negligible functions of $n$.}

\Prop{}{}{For any negligible function $\epsilon$, for any polynomial $P$, the product $P(n)\epsilon(n)$ is still negligible}

\Def{Predictable PRG}{def:predPRG}{
A PRG is \underline{predictable} if there exists a PPT (Proba Poly Time) algorithm $A$ and an index $i\in\llbracket1,m\rrbracket$ such that:
\[\Pr_{k\hookrightarrow U(\{0,1\}^s), \text{$A$'s internal randomness}}(A(G(k|[1...i])=G(k)[i+1])>\frac{1}{2}+\epsilon(s)\]
for some non-negligible function of $s$, $\epsilon>0$.}
\underline{Remark:} It restricts the adversary's power (otherwise $A$ could use all possible $k$) and the probability is taken over the uniform choice of $k$ and the internal randomness of $A$.

\underline{Remark: } A PRG is unpredictable if for all PPT algorithm and index, $\epsilon$ is negligible.

\underline{Question: } How to compute $p:=\Pr_{k\hookrightarrow U(\{0,1\}^s), \text{$A$'s internal randomness}}(A(G(k|[1...i])=G(k)[i+1])$ ?\\
Set $x::=0$.\\
Repeat $N$ times: \begin{itemize}
\item choose $k$ uniformly
\item If $A(G(k|[1...i])=G(k)[i+1]$ increment $x$
\end{itemize}
Output $x/n\approx p$

Using Hoeffding bound we get $\Pr(|x-pN|\geq t)\leq 2\exp(-2t^2/N)$, then $\Pr(|x-pN|\geq t'\sqrt{N})\leq 2\exp(-2t'^2)$ for any $t'$. Then we get $p$ with approximation $1\pm\frac{10}{\sqrt{N}}$ with $t'=10$.


\Def{Distinguishability}{def:disting}{Let $D_1$ and $D_2$ be two distributions. An algorithm $A:\{O,1\}^n\rightarrow\{O,1\}$ with support $\{O,1\}^n$ is distinguishable if $Adv_A(D_1,D_2) := |\Pr_{x\hookrightarrow D_1}(A(x)=1)-\Pr_{x\hookrightarrow D_2}(A(x)=1)|$ is non-negligible.}

$G(\{0,1\}^s)$ is a tiny subset of $\{0,1\}^n$ but we want it to looks uniform. We want it is indistinguishable from the uniform distribution.

\Def{Indistinguishability}{def:indisting}{$D_1$ and $D_2$ are indistinguishable if for any PPT $A:\{0,1\}^n\rightarrow\{0,1\}$ $Adv_A(D_1,D_2)$ is negligible.}

\Def{Pseudo-Random PRG}{def:pseudoRandPRG}{A PRG $G$ is pseudo-random is $D_1:=\{G(k)|k\hookrightarrow U(\{0,1\}^s)\}$ and $D_2:=\{x|x\hookrightarrow U(\{0,1\}^n)\}$ are indistinguishable} 


\end{document}

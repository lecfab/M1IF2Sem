%/!\ /!\ 
%
% PLEASE DO NOT EDIT THIS IF YOU CAME HERE BY MISTAKE !!!!
%

% RTFMN : https://tobi.oetiker.ch/lshort/lshort.pdf

\documentclass{article}
\usepackage{xspace}
\usepackage[utf8]{inputenc}
\usepackage[T1]{fontenc}
\usepackage[francais]{babel}
\usepackage{amsmath}
\usepackage{amsthm}
\usepackage{graphicx}
\usepackage{url}
\usepackage{amssymb}
\usepackage{mathrsfs}
\usepackage{amsfonts}
\usepackage{multicol}
\usepackage{stmaryrd}
\usepackage{tikz, pgf}
\usetikzlibrary{arrows,intersections}
\usepackage{libertine}
\usepackage{cancel}
\usepackage[a4paper,left=2cm,right=2cm,top=2cm,bottom=2cm]{geometry}
\usepackage{dsfont}


\usepackage[linktocpage]{hyperref}

\setlength{\hoffset}{-18pt}         
\setlength{\oddsidemargin}{15pt} % Marge gauche sur pages impaires
\setlength{\evensidemargin}{15pt} % Marge gauche sur pages paires
\setlength{\marginparwidth}{0pt} % Largeur de note dans la marge
\setlength{\textwidth}{481pt} % Largeur de la zone de texte 
\setlength{\marginparsep}{7pt} % Séparation de la marge
\setlength{\topmargin}{0pt} % Pas de marge en haut
\setlength{\headheight}{13pt} % Haut de page
\setlength{\headsep}{10pt} % Entre le haut de page et le texte
\setlength{\footskip}{50pt} % Bas de page + séparation
\setlength{\textheight}{600pt} % Hauteur de la zone de texte 

%\setlength{\hoffset}{-18pt}         
%\setlength{\oddsidemargin}{15pt} % Marge gauche sur pages impaires
%\setlength{\evensidemargin}{15pt} % Marge gauche sur pages paires
%\setlength{\marginparwidth}{0pt} % Largeur de note dans la marge
%\setlength{\textwidth}{481pt} % Largeur de la zone de texte 
%\setlength{\marginparsep}{7pt} % Séparation de la marge
%\setlength{\topmargin}{0pt} % Pas de marge en haut
%\setlength{\headheight}{8pt} % Haut de page
%\setlength{\headsep}{0pt} % Entre le haut de page et le texte
%\setlength{\footskip}{15pt} % Bas de page + séparation
%\setlength{\textheight}{700pt} % Hauteur de la zone de texte 

%\newcommand{\ket}[1]{\ensuremath{|#1\rangle}\xspace}
%\newcommand{\bra}[1]{\ensuremath{\langle #1|}\xspace}

\newtheorem{thm}{Théorème}[section]
\newtheorem{prop}[thm]{Proposition}
\newtheorem{lem}[thm]{Lemme}
\newtheorem{cor}[thm]{Corollaire}
\newtheorem{defi}[thm]{Définition}
\newtheorem{ex}[thm]{Exemple}

\newcommand{\Thm}[3]{\begin{thm}[#1]\label{#2}#3\end{thm}}
\newcommand{\Ex}[3]{\begin{ex}[#1]\label{#2}#3\end{ex}}
\newcommand{\Def}[3]{\begin{defi}[#1]\label{#2}#3\end{defi}}
\newcommand{\Lem}[3]{\begin{lem}[#1]\label{#2}#3\end{lem}}
\newcommand{\Cor}[3]{\begin{cor}[#1]\label{#2}#3\end{cor}}
\newcommand{\Prop}[3]{\begin{prop}[#1]\label{#2}#3\end{prop}}

\newcommand{\hsp}{\hspace{20pt}}
\newcommand{\HRule}{\rule{\linewidth}{0.5mm}}
\newcommand{\R}{\mathbb{R}}
\newcommand{\N}{\mathbb{N}}
\newcommand{\K}{\mathbb{K}}
\newcommand{\Q}{\mathbb{Q}}
\newcommand{\C}{\mathcal{C}}
\newcommand{\X}{\mathcal{X}}
\newcommand{\brakets}[1]{\langle#1\rangle}
\newcommand{\betaar}{\rightarrow_\beta}
\newcommand{\betar}{\twoheadrightarrow_\beta}
\newcommand{\alphaeq}{=_\alpha}

\newcommand{\ind}[1]{\mathds{1}_{#1}}

\title{Preuves et Programmes}
\author{Philippe Audebaud}
\date{Printemps 2017}



\begin{document}

\maketitle

\tableofcontents

\newpage

\section{ Lambda Calcul (pure)}
\subsection{Calculer avec des fonctions (uniquement)}
Main question : How do we do maths ? To answer that we can see:

\begin{itemize}
\item Having structures:
We want to manipulate numbers, spaces (of points, vectors and functions). (Eidenberg-Mac Lane, Category Theory, 1942). Those things will be the \underline{types}.

\item Build, explore and transform structures (Church's $\lambda$-calculus, 1930). Theses things with be programs, proofs.

\item Compare «stuff», with equality for instance (equality comes first !) (Voevaski, 2006, Algebraic Topology)

\item Provide a framework (rules) to reasoning on theses previous things.  This points is somehow the first point that comes when we want to do Maths.
\end{itemize}

One can notice that it is, in fact, basically, recent research.

\subsection{Introduction informelle au $\lambda$-calcul de Church} 

We take a function : $\begin{array}{l c c r}
f: & A & \rightarrow & B\\
& x & \mapsto & e
\end{array}$. Given $a\in A$, $f(a)$ is the image, we replace (kind of) each occurrence of $x$ in $e$ by an occurrence of $a$. Then we get $e\brakets{a/x}$. We say that we apply $f$ to $a$. We can denote $f\ a$ when there is no ambiguity. In terms of $\lambda$-calculus, we can define $f$ like that: $f\equiv \lambda x.e$ where $\equiv$ is the definitional equality. Then $f\ a\equiv (\lambda x.e)\ a$.

For the syntax, if there is no parenthesis, all what is after the $.$ is part of the body of the function.

\Ex{}{}{Here are some examples
\begin{itemize}
\item $\lambda x.x$ is the function $x\mapsto x$ is the identity function.
\item If $x$ and $y$ are two distinct variables the function $\lambda x.y$ as no effect since $y\brakets{a/x}\equiv y$ and then $(\lambda x.y)\ a \equiv y$.
\end{itemize}}

To «compute» terms we have to introduce a kind of reduction $\betaar$, a binary relation on $\lambda$-terms. For instance, $(\lambda x.a) b \betaar a\brakets{b/x}$.

Some terms also seems to be «equivalent». Then we may need some $\lambda$-equivalence, let say $\alphaeq$. For instance we would like to say when $y\neq x$ is a fresh variable that $\lambda x.e \alphaeq \lambda y.e\brakets{y/x}$. To state  $\lambda x.a \alphaeq \lambda y.b$, we might need some fresh variable $z$ such that $a\brakets{z/x}\alphaeq b\brakets{z/y}$.

\Def{$\alpha$-équivalence}{def:alphaEq}{On définit l'$\alpha$-équivalence comme ceci :
\begin{itemize}
\item si $t$ et $x$ sont des variables. $t\brakets{u/x}\alphaeq\left\{\begin{array}{r l}
u & \text{si $t=x$}\\
t & \text{sinon}
\end{array}\right.$
\item $(v\ w)\brakets{u/x}\alphaeq v\brakets{u/x}\ w\brakets{u/x}$
\item $(\lambda x.e)\brakets{u/y}\alphaeq\left\{\begin{array}{r l}
\lambda x.e\brakets{u/y} & \text{si $y\neq x$}\\
t & \text{sinon}
\end{array}\right.$
\end{itemize}}

\subsection{Boite à outils concernant le $\lambda$-calcul.}
\subsubsection{Généralités}
Let $\X$ be a denumerable set of variables range over $x,y,z,...$.
\Def{$\lambda$-terms}{def:lterms}{A $\lambda$-term $e$ is generated by the following grammar $e::= x\in\X | \lambda x.e | e e$. The set of $lambda$ terms is denoted $\Lambda$.}

\Def{Free variable}{def:freeVar}{The set of free variable in a term $e$, denoted $FV(e)$ is	defined inductively by:

\begin{itemize}
\item if $e = x\in\X$ then $FV(x) \equiv \{x\}$.
\item if $e = \lambda x.a$ then $FV(e)\equiv FV(a)\backslash\{x\}$.
\item if $e = a\ b$ then $FV(e)\equiv FV(a)\cup FV(b)$.
\item if $e$ is closed $FV(e)=\emptyset$.
\end{itemize}}

\Ex{}{}{\begin{itemize}
\item $e\equiv \lambda x.x$ then $FV(e) = \emptyset$.
\item $e\equiv \lambda x.y$ then $FV(e) = \{y\}$.
\end{itemize}}

\Def{Subsitution}{def:sub}{Given $x\in\X, a\in\Lambda$ the substitution of (all the) occurrences of $x$ in $e$ by $a$, denoted $e\brakets{a/x}$ is: \begin{itemize}
\item if $e\equiv y\in\X\backslash\{x\}$ then $y\brakets{a/x}\equiv y$ otherwise $x\brakets{a/x}\equiv a$.
\item $(\lambda y.e)\brakets{a/x}\equiv\lambda y.e\brakets{a/x}$
\item $(e\ f)\brakets{a/x}\equiv e\brakets{a/x}\ f\brakets{a/x}$.
\end{itemize} }

\Def{The $\betaar$ relation}{def:betaar}{The binary relation $\betaar$ over $\Lambda$ is $\betaar\subseteq\Lambda\times\Lambda$ such that $\betaar\equiv\{((\lambda x.a)\ b, a\brakets{b/x})|x\in\X,a\in\Lambda,b\in\Lambda\}$}

\Ex{}{}{\begin{itemize}
\item $(\lambda x.(\lambda y.y)\ a)\ b\betaar((\lambda y.y)\ a)\brakets{b/x} \equiv (\lambda y.y)\brakets{b/x}\ a\brakets{b/x} \equiv(\lambda y.y)\ a\brakets{b/x}$
\item $(\lambda x.y)\ a\betaar y$
\item Paradoxe de Russell : $(\lambda x. x\ x) (\lambda x.x\ x)\betaar (x\ x)\brakets{(\lambda x.x\ x)/x}$ or $(x\ x)\brakets{(\lambda y.y\ y)/x} \equiv (\lambda x. x\ x) (\lambda x.x\ x)$ modulo $\alphaeq$ (or exactly that if you consider the case before the "or"). Then it reduces to itself and the reduction does not terminate.
\end{itemize}}

Avec le dernier exemple, on peut voire que l'on a besoin de $\betaar\subseteq\beta_0\subseteq\beta\equiv\beta_0^*$ pour un certain $\beta_0$ et $\beta$ la fameuse $\beta$-réduction, $\beta\equiv\betaar^*$ aussi écrite $\betar$.

\Def{$\beta_0$-contraction}{def:b0contract}{Soit $a,b\in\Lambda$, on définit la relation $a\ \beta_0\ b$ comme ceci : \begin{itemize}
\item $x\ \beta_0\ x$
\item $(\lambda x.u)\ v\ \beta_0\ u\brakets{v/x}$
\item $(\lambda x.u)\ \beta_0\ (\lambda x.v)$ if $u\ \beta_0\ v$
\item $(u\ v)\ \beta_0\ (u'\ v)$ if $u\ \beta_0\ u'$ 
\item $(u\ v)\ \beta_0\ (u\ v')$ if $v\ \beta_0\ v'$ 
\end{itemize}}

\underline{Remarque :} Par induction structurelle on montre que $\beta_0$ est réflexive.

\Def{$\beta$-réduction}{def:betared}{La $\beta$-réduction est la clôture transitive de $\beta_0$, $\beta\equiv\beta_0^*$.}
\underline{Remarque :} Si $a,b\in\Lambda$ alors $a\ \beta\ b$ s'il existe $n\geq 0$ et $(e_k)_{0\leq k\leq n}$ tels que : \begin{itemize}
\item $a\equiv e_0$
\item $b\equiv e_n$
\item $\forall k<n, e_k\ \beta_0\ e_{k+1}$
\end{itemize}

\Def{$\lambda$-compatibilité}{def:lcompat}{Soit $\mathcal{R}$ une relation sur le $\lambda$-calcul, $\Lambda$. On fit que $\mathcal{R}$ est $\lambda$-compatible si : \begin{itemize}
\item elle est réflexive
\item si $a\ \mathcal{R}\ b$ et $c\ \mathcal{R}\ d$ alors $(a\ c)\ \mathcal{R}\ (b\ d)$
\item si $a\ \mathcal{R}\ b$ alors $\lambda x.a\ \mathcal{R}\ \lambda x.b$
\end{itemize}}

\Prop{}{def:betaRedCompat}{La $\beta$-réduction est la plus petite relation transitive $\lambda$-compatible contenant $\betaar$.}

\begin{proof}
\begin{itemize}
\item on vérifie d'abord $\betaar\subseteq\beta_0\subseteq\beta_0^*\equiv\beta$
\item maintenant on vérifie qu'elle est $\lambda$-compatible. \begin{itemize}
\item Elle est bien réflexive
\item Soit $a\ \beta\ b$ et $c\ \beta\ d$. On considère $(e_k)$ et $(f_k)$ associés. Considérons $g_k=\left\{\begin{array}{r l}
(e_k\ f_0) & \text{si $k\leq|e|$}\\
(e_{|e|}\ f_{k-|e|}) & \text{sinon}
\end{array}\right.$. Par définition de $\beta_0$ on obtient le résultat.
\item Soit $a\ \beta\ b$. On considère $(e_k)$ associé. Alors par définition de $\beta_0$ on a $\lambda x.e_k\ \beta\ \lambda x.e_{k+1}$. D'où $\lambda x.a\ \mathcal{R}\ \lambda x.b$
\end{itemize}
\item Soit $\mathcal{R}$ une autre relation $\lambda$-compatible et transitive contenant $\betaar$. Pour cela il suffit de vérifier que $\beta_0\subseteq\mathcal{R}$ et conclure par transitivité.
\end{itemize}
\end{proof}
\subsubsection{Propriétés essentielles de la $\beta$-réduction}
\underline{Remarque : } $(\Lambda,\beta_0)$ est un système de réduction abstrait (voir définition \ref{sysRedAbs}).

\Def{Forme Normale}{def:FormeNormale}{Soit $\mathcal{R}$ une relation binaire sur $\Lambda$. On dit que :
\begin{itemize}
\item $a$ est une forme normale (relativement à $\mathcal{R}$) s'il n'existe pas de $b\in\Lambda$ tel que $a\ \mathcal{R}\ b$
\item $a$ a une forme normale s'il existe une forme normale $b$ telle que $a\ \mathcal{R}^*\ b$.
\item $\mathcal{R}$ est normalisante si $\forall a\in\Lambda, a$ a une forme normale.
\end{itemize}}

\Ex{}{ex:normalisation}{\begin{itemize}
\item $\lambda x.x$ est une forme normale respectivement à $\beta_0$.
\item $\beta_0$ n'est pas normalisante. Par exemple $\Omega\equiv (\lambda x.x\ x)\ (\lambda x.x\ x)$ se réduit vers lui même.
\end{itemize}}

\Def{Confluence}{def:confluence}{Soit $\mathcal{R}$ une relation binaire sur $\Lambda$ si pour tous $a,b,c$ tels que $a\ \mathcal{R}^*\ b$ et $a\ \mathcal{R}^*\ c$ alors il existe $d\in\Lambda$ tel que $b\ \mathcal{R}^*\ d$ et $c\ \mathcal{R}^*\ d$.}
\begin{center}
\begin{tikzpicture}[>=latex]
\node	[]	at	(2,4)	{$a$};
\node	[]	at	(0,2)	{$b$};
\node	[]	at	(4,2)	{$c$};
\node	[]	at	(2,0)	{$d$};
\draw	[->] (1.8,3.8) --node[left] {$\mathcal{R}^*$} (0.2,2.2);
\draw	[->] (2.2,3.8) --node[right] {$\mathcal{R}^*$} (3.8,2.2);
\draw	[->, dashed] (0.2,1.8) --node[left] {$\mathcal{R}^*$} (1.8,0.2);
\draw	[->, dashed] (3.8,1.8) --node[right] {$\mathcal{R}^*$} (2.2,0.2);
\end{tikzpicture}
\end{center}

\Thm{Confulence de $\beta_0$}{thm:beta0Conflu}{$\beta_0$ est confluante.}

\begin{proof}
Cette preuve sera faite plus loin dans le cours.
\end{proof}

\Cor{}{cor:uniciteNormale}{Tout $\lambda$-terme a au plus une forme normale relativement à $\beta_0$}

\begin{proof}
Supposons deux forme normale distinctes $b$ et $c$ pour $a$ alors par confluence on a l'existence de $d$ tel que $b$ et $c$ s'y réduise ce qui est une contradiction.
\end{proof}

\subsubsection{Notion d'égalité sur $\Lambda$.}
\Def{$\lambda$-congruence}{def:lcongru}{Une $\lambda$-congruence et une relation d'équivalence $\lambda$-compatible.}

\Def{$\beta$-équivalence}{def:betaEq}{La $\beta$-équivalence sur $\Lambda$ est la relation binaire $=_\beta$ définie comme la clôture symétrique de $\beta$. On a $a=_\beta b$ s'il existe $(e_k)_{0\leq k\leq n}$ tel que $e_0=a$ et $e_n=b$ et pour tout $k<n$ $e_k\ \beta_0\ e_{k+1}$ ou $e_{k+1}\ \beta_0\ e_k$.}

\Thm{Chuch-Rosser}{thm:ChurchRosser}{Pour tous $a,b\in\Lambda$, $a=_\beta b$ ssi il existe $c\in\Lambda$ tel que $a\ \beta\ c$ et $b\ \beta\ c$.}
\begin{proof}
\begin{itemize}
\item La relation est clairement suffisante.
\item Pour la condition nécessaire, on introduit $\mathcal{R}\subseteq\Lambda^2$ définit par $a\ \mathcal{R}\ b$ ssi il existe $c\in\Lambda$ tel que $a\ \mathcal{R}\ c$ et $b\ \mathcal{R}\ c$. Elle est réflexive et symétrique. On vérifie encore qu'elle est transitive grâce à la confluence de $\beta$.
\begin{center}
\begin{tikzpicture}[>=latex]
\node	[]	at	(0,4)	{$a$};
\node	[]	at	(2,4)	{$b$};
\node	[]	at	(4,4)	{$c$};
\node	[]	at	(1,2)	{$d_1$};
\node	[]	at	(3,2)	{$d_2$};
\node	[]	at	(2,0)	{$e$};
\draw	[->] (1.8,3.8) --node[left] {$\mathcal{R}^*$} (1.2,2.2);
\draw	[->] (2.2,3.8) --node[right] {$\mathcal{R}^*$} (2.8,2.2);
\draw	[->] (0.2,3.8) --node[left]  {$\mathcal{R}^*$} (0.8,2.2);
\draw	[->] (3.8,3.8) --node[right]  {$\mathcal{R}^*$} (3.2,2.2);
\draw	[->, dashed] (1.2,1.8) --node[left] {$\mathcal{R}^*$} (1.8,0.2);
\draw	[->, dashed] (2.8,1.8) --node[right] {$\mathcal{R}^*$} (2.2,0.2);
\end{tikzpicture}
\end{center}

C'est donc une relation d'équivalence qui contient $\beta$. Par définition \ref{def:betaEq} on conclut à l'égalité (plus petite relation).
\end{itemize}
\end{proof}

\Thm{}{thm:betaEqpluspetite}{$=_\beta$ est la plus petite $\lambda$-congruence contenant $\betaar$}
\begin{proof}
Exercice
\end{proof}

\section{Calcul propositionnel}
\subsection{Éléments de langage (informel)}
On parle ici de théorie de la démonstration (prouvabilité) et la théorie des modèles (validité, vérité). On parle donc du lien entre correction (ce qui est prouvé est vrai) et complétude (ce qui est vrai est prouvable). 

\Def{Proposition}{def:prop}{Les formules du calcul propositionnel sont définies par la grammaire suivante :
\[A::=X|\top|\bot|A\Rightarrow B|A\wedge B| A\vee B\cancel{|\neg A}\]
On dira alors que $A$ engendré par cette grammaire est une proposition.}

On portera des jugement sur ces énoncés (propositions) du genre "$A$ $true$". On aura aussi des des jugements hypothétiques $A_1\ true,..., A_n\ true\vdash B\ true$.
\end{document}
